\documentclass[a4paper]{article} %twocolumn
\usepackage[utf8]{inputenc}
\usepackage{graphicx}
\usepackage{listings}
\usepackage{fancyhdr}
\usepackage{titlesec}
\usepackage{amsmath}
\usepackage{color}
\usepackage{hyperref}
\usepackage{pdfpages}
\pagestyle{fancy}

\title{TMA4280 Super Computers} 
\author{Jakob Hovland\\Jørgen Grimnes\\Assignment 4}
\date{Spring 2015}

\lhead{Jakob Hovland Jørgen Grimnes\\Spring 2015}
\rhead{TMA4280 Super Computers\\Assignment 4}
\renewcommand{\headrulewidth}{0.4pt}
\renewcommand{\footrulewidth}{0.4pt}

% downsize title
%\titleformat{\section}{\large\bfseries}{\thesection}{1em}{}


\begin{document}
	\maketitle
   \pagebreak
   
   \section{Introduction}
   This report presents a solution to homework project number four in TMA4280 Super Computers, spring 2015, at Norwegian University of Science and Technology. The assignment is a introduction to implementing efficient and parallel programs.
   
   \section{C++ program}
   Please see the file \emph{summation.cpp}
      
   \section{OpenMP}
   \subsubsection*{Do the necessary changes to utilize shared memory parallelization through
OpenMP}
   Please see the file \emph{summation openmp.cpp}
   
   \section{MPI program}
   \subsubsection*{Write a program to compute the sum $S_n$ using P processors where P is a power of 2, and a distributed memory model (MPI)}
   Please see the file \emph{summation mpi.cpp}.\\
   Report the difference $S - S_n$ in double precision for different values of $n$.
   
   \section{OpenMP and MPI}
   \subsubsection*{Confirm that your program also works when you are using OpenMP and MPI in combination}
   Please see the file \emph{summation combined.cpp} for confirmation.
   
   \section{Which MPI calls are convenient/necessary to use?}

\begin{table}[hbt]
  \begin{center}
    \begin{tabular}{|l|p{20em}| }
      \hline
      \textbf{Call} & \textbf{Description}\\
      \hline
      MPI::Init & Initializes the MPI execution environment\\ \hline
      MPI::COMM\_WORLD.Get\_size & Returns the size of the group associated with COMM\_WORLD. \\ \hline
      MPI::COMM\_WORLD.Get\_rank & Determines the rank of the calling process in COMM\_WORLD.\\ \hline
      MPI::COMM\_WORLD.Bcast & Broadcasts a message from the process with rank root to all other processes of the group.\\ \hline
      MPI::COMM\_WORLD.Reduce & Reduces values on all processes within COMM\_WORLD\\ \hline
      MPI::Finalize & Terminates MPI execution environment. \\
      \hline
    \end{tabular}
  \end{center}
  \caption{Non-Functional Requirements}
\end{table}
   
\iffalse\begin{figure}[ht!]
\center \includegraphics[width=280px]{plot.pdf}
\end{figure}\fi

\end{document}
