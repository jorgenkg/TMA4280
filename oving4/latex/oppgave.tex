\documentclass[a4paper]{article} %twocolumn
\usepackage[utf8]{inputenc}
\usepackage{graphicx}
\usepackage{listings}
\usepackage{fancyhdr}
\usepackage{titlesec}
\usepackage{amsmath}
\usepackage{color}
\usepackage{hyperref}
\usepackage{pdfpages}
\pagestyle{fancy}

\title{TMA4280 Super Computers} 
\author{Jakob Hovland\\Jørgen Grimnes\\Assignment 4}
\date{Spring 2015}

\lhead{Jakob Hovland Jørgen Grimnes\\Spring 2015}
\rhead{TMA4280 Super Computers\\Assignment 4}
\renewcommand{\headrulewidth}{0.4pt}
\renewcommand{\footrulewidth}{0.4pt}

% downsize title
%\titleformat{\section}{\large\bfseries}{\thesection}{1em}{}


\begin{document}
\maketitle
\pagebreak

\section{Introduction}
This report presents a solution to homework project number four in TMA4280 Super Computers, spring 2015, at Norwegian University of Science and Technology. The assignment is a introduction to implementing efficient and parallel programs.

\section{Write a program (either in C or in FORTRAN) which will do the following}
\begin{itemize}
\item generate the vector $v$
\item compute the sum $S_n$ in double precision on a single processor
\item compute the difference $S-S_n$ for $n=2^k$, with $k=3,..,14$
\item print out the difference $S - S_n$ in double precision.
\end{itemize}
The implemented program was written in C++ and tested on a intel compiler. The lightweight program of only about 25 SLOC uses the build-in C++ vector data structure.
\begin{table}[hbt]
  \begin{center}
    \begin{tabular}{|l|l||l|l| }
      \hline
      \textbf{k} & \textbf{$S - S_n$} & \textbf{k} & \textbf{$S - S_n$}\\
      \hline
      3 & 0.11751 & 9 & 0.0019512\\    
      4 & 0.060588 & 10 & 0.00097609\\ 
      5 & 0.030767 & 11 & 0.00048816\\ 
      6 & 0.015504 & 12 & 0.00024411\\ 
      7 & 0.0077821 & 13 & 0.00012206\\
      8 & 0.0038986 & 14 & 6.1033e-05\\
      \hline
    \end{tabular}
  \end{center}
  \caption{Error measurements for the linear program}
\end{table}


\section{Do the necessary changes to utilize shared memory parallelization through OpenMP}
We included OpenMP parallelization by using the compiler pragma \emph{omp parallel for}. Please see refer to the file \emph{summation openmp.cpp}

\section{MPI program}
\subsubsection*{Write a program to compute the sum $S_n$ using $P$ processors where $P$ is a power of $2$, and a distributed memory model (MPI). The program should work as follows: Only processor $0$ should be responsible for generating the vector elements. Processor $0$ should partition and distribute the vector elements evenly among all the processors. Each processor should be responsible for summing up its own part. At the end, all the partial sums should be added together and made available on processor $0$ for printout. Report the difference $S - S_n$ in double precision for different values of $n$.}
Please see the file \emph{summation mpi.cpp}.\\
   
\section{Confirm that your program also works when you are using OpenMP/MPI in combination.}
Please see the file \emph{summation mpi openmp.cpp} for compiling example of using MPI and OpenMP in combination.

\section{Which MPI calls are convenient/necessary to use?}
We have been operating with the \emph{MPI::COMM\_WORLD} communication group in this project.
\begin{table}[hbt]
  \begin{center}
    \begin{tabular}{|l|p{20em}| }
      \hline
      \textbf{Call} & \textbf{Description}\\
      \hline
      MPI::Init & Initializes the MPI execution environment\\ \hline
      MPI::COMM\_WORLD.Get\_size & Returns the size of the group associated with COMM\_WORLD. \\ \hline
      MPI::COMM\_WORLD.Get\_rank & Determines the rank of the calling process in COMM\_WORLD.\\ \hline
      MPI::COMM\_WORLD.Send & Performs a standard-mode blocking send.\\ \hline
      MPI::COMM\_WORLD.Recv & Performs a standard-mode blocking receive.\\ \hline
      MPI::COMM\_WORLD.Reduce & Reduces values on all processes within COMM\_WORLD\\ \hline
      MPI::Finalize & Terminates MPI execution environment. \\
      \hline
    \end{tabular}
  \end{center}
  \caption{Executed MPI calls}
\end{table}
   
   
\section{Compare the difference $S - S_n$ from the single-processor program and the multi-processor program when $P = 2$ and when $P = 8$. Should the answer be the same in all these cases?}
The error rate of our calculation $S - S_n$ will increase as we distribute our program over an increasing number of processors due to the floats and rounding error. This is clearly visible in \autoref{tbl:error2} and \autoref{tbl:error8}. The tables compares the error measurments between the linear and the dual/octo processed programs as we increase k.


\begin{figure}[ht!]
\center \includegraphics[width=180px]{error_plot.pdf}
\caption{Comparison of the error reduction}
\end{figure}

\begin{table}[hbt]
  \begin{center}
    \begin{tabular}{|l|l|l||l|l|l| }
      \hline
      \textbf{k} & \textbf{$error_{P=2}$} & \textbf{$error_{P=1}$} & \textbf{k} & \textbf{$error_{P=2}$} & \textbf{$error_{P=1}$}\\
      \hline
      3 & 0.13314   & 0.11751   & 9 & 0.001955    & 0.0019512  \\ 
      4 & 0.064494  & 0.060588  & 10 & 0.00097704 & 0.00097609 \\ 
      5 & 0.031743  & 0.030767  & 11 & 0.0004884  & 0.00048816 \\ 
      6 & 0.015748  & 0.015504  & 12 & 0.00024417 & 0.00024411 \\ 
      7 & 0.0078431 & 0.0077821 & 13 & 0.00012208 & 0.00012206 \\ 
      8 & 0.0039139 & 0.0038986 & 14 & 6.1037e-05 & 6.1033e-05 \\
      \hline
    \end{tabular}
  \end{center}
  \caption{Error measurements for $P = 2$} \label{tbl:error2}
\end{table}

\begin{table}[hbt]
  \begin{center}
    \begin{tabular}{|l|l|l||l|l|l| }
      \hline
      \textbf{k} & \textbf{$error_{P=8}$} & \textbf{$error_{P=1}$} & \textbf{k} & \textbf{$error_{P=8}$} & \textbf{$error_{P=1}$}\\
      \hline
      3 & 0.64493   & 0.11751   & 9  & 0.00208  & 0.0019512  \\ 
      4 & 0.19244   & 0.060588  & 10 & 0.0010083 & 0.00097609 \\ 
      5 & 0.063731  & 0.030767  & 11 & 0.00049621 & 0.00048816 \\ 
      6 & 0.023745  & 0.015504  & 12 & 0.00024612 & 0.00024411 \\ 
      7 & 0.0098423 & 0.0077821 & 13 & 0.00012257 & 0.00012206 \\ 
      8 & 0.0044137 & 0.0038986 & 14 & 6.1159e-05 & 6.1033e-05 \\
      \hline
    \end{tabular}
  \end{center}
  \caption{Error measurements for $P = 8$} \label{tbl:error8}
\end{table}

\iffalse\begin{figure}[ht!]
\center \includegraphics[width=280px]{plot.pdf}
\end{figure}\fi

\end{document}
